
Due to artificial intelligence (AI) being a continuously developing field of research, which in the previous years 
has shown a great level of growth, environments to develope and train these AI techniques are crucial.
A popular technique to test the capabilities of AI algorithms, is to 
expose them to games with the goal of playing the game in such a way that the probability of 
winning is maximized. Games which have been optimized on include chess\cite{CAMPBELL200257}, DOTA 2\cite{SilverHuangEtAl16nature}
and most recently GO\cite{openai2019dota}. \par
Games constitute great environments to train and test AI algorithms in since they are
constrained by a rigid predicable rule set, which allows for easy simulation and scaling. \par
In this paper a genetic algorithm (GA) is applied to the game Ludo, in order to generate 
the best possible player through continual evolution and natural selection. 
As a global optimization algorithm this method will be compared to a similar technique which uses 
binary chromosome representation\cite{peter}, rather than the floating point chosen for this paper. \par
To compare these algorithms this paper aims to first determine an optimal GA configuration using a subset 
of hyperparameters, then compare said configuration to the corresponding best representative from the binary 
chromosome representation paper\cite{peter}.\par
The training done in this paper is performed on a AMD Ryzen 7 4700U with Radeon Graphics 2 GHz.

% \begin{itemize}
% 	\item board games provide a controlled and predictable environment for testing of relatively simple AI and GA algorithms.
% 	\item One example of this is the Ludo game which we want to solve. The method of choice is..
% 	\item the method of evolution has stood the test of time, and as a global optimization algorithm, it is chosen
% 	\item We want to compare the performance of GA to an GA algorithm (Peter's DRL)
% 	\item floats vs bits
% \end{itemize}

% Here you motivate the problem and describe what kind of methods you want to use and
% why!.  In a real conference paper, this would be longer because the authors have to 
% explain to other people why the problem is interesting.  
% Here you are working with Ludo because that is the assigned task.  
% You should introduce the content of your paper, but you need n't 
% explain why Ludo is an interesting problem.